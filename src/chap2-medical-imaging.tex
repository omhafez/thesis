\chapter{Medical Imaging}
%
\begin{figure}[tbh]
\centering
\includegraphics{media/ucdavisthesis_example_figure}
% where an .eps filename suffix will be assumed under latex,
% and a .pdf suffix will be assumed for pdflatex
\caption[First sample figure]{A sample figure.}
\label{fig.sample_1}
\end{figure}

\section{Magnetic Resonance Imaging}
\label{Magnetic Resonance Imaging}

MRI is ideal for visually distinguishing soft tissues, but objects MUST contain hydrogen molecules (e.g. water). However it is possible to get around this problem and scan “dry” objects made of plastic, for example, by immersing them in jelly. The negative of the object is then visible in the MRI data. Segmentation can be threshold based in some cases. Unfortunately it is quite common for different objects to be easily distinguishable visually, by texture, but not by greyscale. In these cases some level of manual segmentation may be required. MRI images often suffer from signal attenuation and/or noise on the borders of the region of interest. \\

MRI, CT, ultrasound, others \\

%%%%%%%%%%%%%%%%%%%%%%%%%%%%%%%%%%%%%%%%%%%%%%%
%%%%%%%%%%%%%%%%%%%%%%%%%%%%%%%%%%%%%%%%%%%%%%%
\subsection{Diffusion Tensor MRI}
\label{Diffusion Tensor MRI}

\section{Computed Tomography}
\label{Computed Tomography}
X-ray computed tomography (CT) $\rightarrow$ becoming more popular \\
%%%%%%%%%%%%%%%%%%%%%%%%%%%%%%%%%%%%%%%%%%%%%%%
%%%%%%%%%%%%%%%%%%%%%%%%%%%%%%%%%%%%%%%%%%%%%%%
\section{Other Imaging Technologies}
\label{Other Imaging Technologies}
Ultrasound (3D), Elastography, etc.