\chapter{Physics-Based Modeling and Simulation}
\label{chap:4}
%
REFERENCES:
\cite{rashid_2017}
\cite{rashid_201}
\cite{rashid_212, rashid_289, dafalias_205, suku_212}
jaumann rate \cite{rashid_1991}
incremental kinematics \cite{rashid_1993}
incremental objectivity \cite{rashid_1996}

An explicit geometric description of the biological tissues of interest may be utilized to model a wide array of physical phenomena via several different numerical approaches. For these purposes, the discussion will be restricted to modeling quasistatic finite-deformation nonlinear solid mechanics behavior via finite element methods. Selected relevant topics in nonlinear solid mechanics and conventional finite element methods are presented, followed by a brief summary of a polyhedral finite element method that is to be explored as an alternative approach in modeling the mechanics of biological tissues.

%%%%%%%%%%%%%%%%%%%%%%%%%%%%%%%%%%%%%%%%%%%%%%%
\section{Nonlinear Solid Mechanics}
\label{Nonlinear Solid Mechanics}

Pertinent topics of nonlinear solid mechanics will be briefly summarized herein. Biological tissues can easily experience strains of 30$\%$ or more, so finite deformation kinematics of material bodies will be considered. Although dynamics are of critical importance for many biomechanical studies, the formulation will be restricted here to problems in which inertial terms are considered negligible.

%%%%%%%%%%%%%%%%%%%%%%%%%%%%%%%%%%%%%%%%%%%%%%%
\subsection{Kinematics Preliminaries}

The specification of \textit{nonlinear} solid mechanics is due to \textit{geometric nonlinearities} and/or \textit{material nonlinearities}. Contact considerations may also produce nonlinearities, but will not be considered here. Geometric nonlinearities arise due to the assumed \textit{finite} deformations that occur for a body $\mathcal{B}$ of interest. Unlike small deformation solid mechanics, in the case of finite deformations the fixed \textit{reference configuration} $\kappa_0$ and \textit{current configuration} $\kappa_t$ are considered distinct. The reference configuration is typically chosen as the unloaded state of the body, defined at time $t=0$. Material points in the reference configuration are defined by the position vector $\bm{X}$, and locations in the current configuration by $\bm{x}(\bm{X},t)$. The displacement $\bm{u}(\bm{X},t)$ at a material point is $\bm{u}(\bm{X},t) = \bm{x}(\bm{X},t) - \bm{X}$.

The fundamental kinematic quantity in nonlinear solid mechanics is the \textit{deformation gradient} $\bm{F} = \frac{\partial \bm{x}}{\partial \bm{X}}$, which maps infinitesimal line segments in the reference configuration $\bm{dX}$ to infinitesimal line segments in the current configuration $\bm{dx}$, namely: $\bm{dx} = \bm{F}\bm{dX}$. Provided the reference configuration and deformation gradient, the current configuration of the deformed body is fully defined. The local volume change at a particular point in $\mathcal{B}$ is defined by the determinant of $\bm{F}$, known as the \textit{Jacobian}: $J = \det{\bm{F}}$.

The mapping from reference configuration to current configuration $\bm{F}$ may be decomposed into a  stretch followed by a pure rotation, or vice versa, in what is known as a \textit{polar decomposition}:
\begin{equation}
\bm{F} = \bm{R}\bm{U} = \bm{V}\bm{R}
\end{equation}
where $\bm{U}$ and $\bm{V}$ are symmetric tensors, and $\bm{R}$ is an orthonormal tensor.

Several strain measures may be defined in terms of the deformation gradient. The reader is referred to Chadwick~\cite{chadwick_1999} for a discussion of those definitions.

Although we will focus on quasistatics problems, numerical solutions are still obtained incrementally, and thus rate quantities derived from $\bm{F}$ also come in use. The \textit{rate of deformation} (or stretch rate) tensor $\bm{D}$ and \textit{vorticity} tensor $\bm{W}$ are defined as the symmetric and antisymmetric parts of the spatial \textit{velocity gradient} $\bm{L} = \frac{\partial \bm{v}}{\partial \bm{x}}$.

These kinematic quantities will be heavily utilized in developing the governing equations of nonlinear solid mechanics, developing kinematic update strategies to progress the solution, and defining material models and boundary conditions. A more detailed account of them can be found in Chadwick~\cite{chadwick_1999} and Dafalias~\cite{dafalias_205}.

%%%%%%%%%%%%%%%%%%%%%%%%%%%%%%%%%%%%%%%%%%%%%%%
\subsection{Traction and Stress Measures}

We define the \textit{Cauchy traction} $\bm{t}$ at a particular location on a surface in the following manner:
\begin{equation}
\bm{t} = \lim_{\substack{{da} \rightarrow 0, \\ {\bm{n} fixed}}} \frac{{\bm{df}}}{da}
\end{equation}
where $\bm{da} = \bm{n}da$ is a differentially small unit of area in the current configuration, and $\bm{n}$ is the corresponding normal to that current-configuration surface. The term $\bm{df}$ is a differentially small force in the current configuration acting on $\bm{da}$. Thus, the Cauchy traction is defined as the \textit{current configuration} force per \textit{current configuration} area at a given point acting on a given surface. Applying a balance of linear momentum on a tetrahedral body yields the following relationship: $\bm{t} = \bm{T}^T\bm{n}$. The quantity $\bm{T}$ is the \textit{Cauchy stress} tensor, which defines the \textit{current configuration} force per \textit{current configuration} area at a particular location in space \textit{for any orientation}. It is the most common measure of stress. The Cauchy stress tensor is symmetric due to conservation of angular momentum for arbitrary portions of a body of interest, and so the more common relationship between Cauchy traction and Cauchy stress is as follows:
\begin{equation}
\bm{t} = \bm{T}\bm{n}
\end{equation}
Other common measures of stress are the \textit{first Piola-Kirchhoff stress} $\bm{P}$ and \textit{second Piola-Kirchhoff stress} $\bm{S}$. The first P-K stress relates forces in the \textit{current configuration} to areas in the \textit{reference configuration}, and the second P-K stress relates forces in the \textit{reference configuration} to areas in the \textit{reference configuration}. The first P-K stress is not symmetric. If we define the normal in the reference configuration as $\bm{N}$, we may define the \textit{Piola traction} as $\bm{p} = \bm{P}\bm{N}$ in an analogous manner as was done for Cauchy stress. Cauchy and Piola tractions act in the same direction since they are both defined in a manner in which forces  are in the current configuration. They are related based on static equivalency of forces: $\bm{p}dA = \bm{t}da$, and thus $\bm{p} = \alpha\bm{t}$, where $\alpha = \frac{da}{dA} = J[\bm{N} \cdot \bm{F}^{-1}\bm{F}^{-T}\bm{N}]^{1/2}$ is the \textit{area ratio}.

The first P-K stress, second P-K stress, and Cauchy stress are related to one another in the following manner:
\begin{align}
\bm{P} &= J\bm{T}\bm{F}^{-T} \\
\bm{P} &= \bm{F}\bm{S}
\end{align}
which arise from utilizing the area ratio and deformation gradient to move between reference and current configurations for the differential force and differential area.

%%%%%%%%%%%%%%%%%%%%%%%%%%%%%%%%%%%%%%%%%%%%%%%
\subsection{Governing Equations}
Balance of linear momentum for the body $\mathcal{B}$ in the \textit{reference configuration} amounts to the following relationship:
\begin{equation}
\int_{\partial \kappa_0}\bm{p}dA + \int_{\kappa_0}\rho_0\bm{b}dV = \bm{0}
\end{equation}
where $\partial \kappa_0$ is the boundary of the body $\mathcal{B}$ in the reference configuration, $\rho_0$ is the mass per unit reference configuration volume, $\bm{b}(\bm{X})$ is a body force, and inertial terms are deemed negligible. By making use of the relationship $\bm{p} = \bm{P}\bm{N}$ and the divergence and localization theorems, the final governing equations for nonlinear solid mechanics in index notation are:
\begin{gather}
P_{ij,j} + \rho_0b_i = 0, \text{\ \ } \forall \text{\ \ } \bm{X} \in \kappa_0 \label{eqn:equilibrium} \\
{u_i} = \overline{{u}}_i \text{\ \ on\ \ } \partial_u\kappa_0 \\
{p_i} = \overline{{p}}_i\text{\ \ on\ \ } \partial_t\kappa_0
\end{gather}
where $(),_j = \frac{\partial()}{\partial {{X_j}}}$ are derivatives with respect to the reference-configuration position, $\overline{{u}}_i$ are prescribed displacement boundary conditions on a subset $\partial_u\kappa_0$ of the reference configuration boundary, ${\overline{p}_i}$ are prescribed Piola tractions on a subset $\partial_t\kappa_0$ of the reference configuration boundary, $\partial_u\kappa_0 \text{\ }\bigcup\text{\ } \partial_t\kappa_0 = \partial\kappa_0$, and $\partial_u\kappa_0 \text{\ }\bigcap\text{\ } \partial_t\kappa_0 = \emptyset$. Robin boundary conditions are omitted here for clarity. Balance of linear momentum is enforced in the \textit{reference} configuration for reasons to be explained in the next section. The field equations above for nonlinear solid mechanics become a well-posed elliptic boundary value problem when combined with the definition of the deformation gradient and a constitutive model, and is known as the \textit{strong form} of the problem statement.

%%%%%%%%%%%%%%%%%%%%%%%%%%%%%%%%%%%%%%%%%%%%%%%
%%%%%%%%%%%%%%%%%%%%%%%%%%%%%%%%%%%%%%%%%%%%%%%
\section{The Finite Element Method in Computational Solid Mechanics}
\label{The Finite Element Method in Computational Solid Mechanics}

Finite element methods are the most popular numerical methods for solving nonlinear solid mechanics problems. Finite element approximations are based on the \textit{weak}, or variational problem statement. The weak form of the problem statement is presented, followed by a brief discussion of Galerkin approximations to the weak form of equilibrium, of which finite element methods are a special case~\cite{rashid_2017}. Some of the major considerations for conventional FEM are discussed, including solution of the the nonlinear residual equations, mesh considerations, constitutive models, and boundary conditions.

\subsection{Weak Form and Galerkin Approximation}

The weak form of \eqnref{equilibrium} enforces equilibrium in an average integral sense, in which the equations are dotted with a \textit{test function} $\bm{w}$ and integrated over the body $\mathcal{B}$ in the reference configuration:
\begin{equation}
\int_{\kappa_0}{P_{ij,j}}w_idV + \int_{\kappa_0}\rho_0b_iw_idV = 0, \text{\ \ } \forall \text{\ \ } \bm{w} \in \mathcal{V}
\end{equation}
where $\mathcal{V} = \lbrace\bm{w} \text{\ }|\text{\ } \bm{w} \in H^1(\kappa_0), \text{\ }\bm{w} = 0 \text{\ on\ } \partial_u\kappa_0\rbrace$.

Define a \textit{trial solution space} $\mathcal{S} = \lbrace\bm{u} \text{\ }|\text{\ } \bm{u} \in H^1(\kappa_0), \text{\ }\bm{u} = \overline{\bm{u}} \text{\ on\ } \partial_u\kappa_0\rbrace$  to which displacements $\bm{u}$ must belong. Utilizing the divergence theorem, the weak form of equilibrium takes the following form:

Find $\bm{u}(\bm{X},t) \in \mathcal{S}$ such that
\begin{gather}
\int_{\kappa_0}P_{ij}w_{i,j}dV - \int_{\kappa_0}\rho_0b_iw_idV - \int_{\partial_t\kappa_0}\overline{p}_iw_idA = 0 \text{\ \ } \forall \text{\ \ } \bm{w} \in \mathcal{V}
\end{gather}

The weak form is solved via Galerkin approximation, provided the following approximations to $\bm{u}$ and $\bm{w}$ via a linear combination of a finite set of \textit{basis functions} $\phi_a$:
\begin{gather}
\bm{u}(\bm{X},t) = \sum\limits_{a}\bm{u}_a(t)\phi_a(\bm{X}) \\
\bm{w}(\bm{X},t) = \sum\limits_{a}\bm{w}_a(t)\phi_a(\bm{X})
\end{gather}
where $\bm{u}_a$ and $\bm{w}_a$ are nodal values. The approximation is of the Galerkin variety because the basis functions that approximate $\bm{u}$ and $\bm{w}$ are the same. A finite discretization of $\mathcal{B}$ provides the nodes and elements that approximate the geometry of the body, and help define the basis functions. The nodes $a$ consist of the all those in the discretization that do not lie on $\partial_a\kappa_0$. We assume here and henceforth homogeneous displacement boundary conditions for simplicity. The basis functions are globally continuous functions that exhibit important features in a finite element setting, namely: 1) \textit{compact support}, meaning $\phi_a$ is nonzero only on elements that contain node $a$; 2) the \textit{Kronecker delta} property, meaning $\phi_a(\bm{X}_b) = \delta_{ab}$, where $b$ is another node in the reference configuration; and 3) \textit{partition of unity}, meaning $\sum\limits_a\phi_a(\bm{X}) = 1$ throughout the body.

Utilization of the approximations to $\bm{u}$ and $\bm{w}$, and the fact that equilibrium must hold for any choice of $\bm{w}_a$ as long as $\bm{w} \in \mathcal{V}$, lead to the finite element approximation to the weak form of equilibrium, known as the \textit{residual equations}:
\begin{equation}
\bm{R}_a = \int_{\kappa_0}P_{ij}\phi_{a,j}dV - \int_{\kappa_0}\rho_0b_i\phi_adV - \int_{\partial_t\kappa_0} \overline{p}_i\phi_adA = \bm{0}
\label{eqn:elresid}
\end{equation}
Shape functions and integrals are all performed in the \textit{reference configuration}, making this a \textit{total Lagrange} formulation. Performing all integrals in the reference configuration allows the basis function derivatives to be stored, rather than recomputed for each iteration of each solution step. The global residual equations are a nonlinear system of equations, solved in steps to ensure equilibrium is enforced throughout the quasistatic loading.

\subsection{Solution of the Nonlinear Residual Equations}

The global residual equations are solved in incremental steps, progressing the solution from one equilibrium state to the next until the final equilibrium state is reached for the quasistatic problem. Assume equilibrium has been established for time $t_n$ and we seek to find equilibrium for the next time step to time $t_{n+1}$, where $\Delta{t} = t_{n+1} - t_{n}$. The \textit{beginning step} will refer to $t_n$ and the \textit{end step} to $t_{n+1}$. Define the positions $\overline{\bm{x}} = \bm{x}(\bm{X},t_n)$, and for the forthcoming purposes we also refer to ${\bm{x}}$ as $\bm{x}(\bm{X},t_{n+1})$. Define ${\bm{u}}^* = \overline{\bm{x}} - \bm{X}$. Thus, in order to solve for displacements $\bm{u} = \bm{x} - \bm{X}$ at time $t_{n+1}$, we seek to find the incremental nodal displacements $\hat{\bm{u}} = \bm{x} - \overline{\bm{x}}$ such that $\bm{u}$ is updated via $\bm{u} = {\bm{u}}^* + \hat{\bm{u}}$. Thus, when seeking to update from time $t_n$ to $t_{n+1}$, we seek the incremental displacements $\hat{\bm{u}}$ such that the global residual equations are satisfied at time $t_{n+1}$ within a tolerance.

The most common approach to solving the nonlinear system of equations is via a Newton-Rapshon iterative scheme. The solution of $\bm{R}_a(t_{n+1}) = \bm{0}$ is performed via successive solution of the following \textit{linear} global stiffness equations:
\begin{equation}
\frac{\partial R_{ia}}{\partial \hat{u}_{jb}} \bigg|_k \delta\hat{u}_{jb} = -R_{ia}\bigg|_k
\label{eqn:newtonraphson}
\end{equation}
where $k$ is the iteration number, and $\delta\hat{u}_{jb}$ is the correction to the incremental nodal displacement. When applied successfully, the procedure corrects the incremental nodal displacements in a manageable number of iterations (ideally on the order of 5, and hopefully fewer than 20 or fewer in practice) until they satisfy that $\bm{R}_a(t_{n+1}) = \bm{0}$.

Populating and solving the global stiffness equations requires computing the following two terms: $\frac{\partial R_{ia}}{\partial \hat{u}_{jb}} \big|_k$ and $R_{ia}\big|_k$. The nodal residuals have been defined previously in \eqnref{elresid}. For a constant reference density, and assuming the body force is not a function of displacement, the tangent stiffness is:
\begin{equation}
\frac{\partial R_{ia}}{\partial \hat{u}_{jb}} = \int_{\kappa_0}\frac{\partial P_{ik}}{\partial \hat{u}_{jb}}\phi_{a,k}dV - \int_{\partial_t\kappa_0}\frac{\partial \overline{p}_i}{\partial \hat{u}_{jb}}\phi_adA
\end{equation}
The manner in which the integrals in these terms are performed is based on the discretization of the body $\mathcal{B}$ into a discrete \textit{mesh} of finite elements. The calculation of $P_{ij}$ and $\partial{P}_{ik}/\partial{\hat{u}_{jb}}$ relies on a careful treatment of the \textit{constitutive update} and corresponding material model. The calculation of $\overline{p}_i$ and $\partial{\overline{p}_i}/\partial{\hat{u}_{jb}}$ corresponds to the choice of traction \textit{boundary conditions}. Each of these three important components to FEM are discussed next.

%%%%%%%%%%%%%%%%%%%%%%%%%%%%%%%%%%%%%%%%%%%%%%%
\subsection{Mesh Considerations}

The integrals in the global stiffness equations are performed on an element level and subsequently combined to form the global residual and tangent stiffness. Integrals over a single element domain $\Omega_e \subseteq \kappa_0$ have the same form as those defining $\frac{\partial R_{ia}}{\partial \hat{u}_{jb}} \big|_k$ and $R_{ia}\big|_k$, with the exception that integrals are performed over $\Omega_e$ instead of $\kappa_0$. Thus, for each element, the quantities $\frac{\partial R^e_{ia}}{\partial \hat{u}_{jb}} \big|_k$ and $R^e_{ia}\big|_k$ are of interest.

Via the spatial discretization of $\mathcal{B}$ into \textit{elements}, basis functions are decomposed into element \textit{shape functions} defined on each of the neighboring elements to the node $a$ for which $\phi_a$ is defined. Those shape functions are used to compute volume and facet integrals via an \textit{isoparametric transformation} from the physical space to the parent space, in the following manner:
\begin{equation}
\bm{X}(\bm{\xi}) = \sum\limits_{a}\bm{X}_a{N}_a(\bm{\xi})
\end{equation}
where $\bm{X}_a$ are spatial coordinates of the element nodes, $N_a$ are the shape functions, and $\bm{\xi}$ is the position vector in the parent element domain. The transformation is isoparametric because the same shape functions, and thus basis functions, are used to interpolate nodal displacements. 

Refer to Hughes~\cite{hughes_2007} for a definition of the shape functions $N_a$ for the linear and quadratic tetrahedral and hexahedral elements. The order of an element corresponds to the polynomial degree of the shape functions, as well as the subsequent order of approximation of the displacement solution. Because stress and strain involve the spatial derivative of displacement, stress and strain are approximated one degree lower than is displacement. 

Volume and facet integrals are evaluated via numerical quadrature rules on the parent element and subsequently \textit{assembled} into the global stiffness equations. For conventional FEM, the quality of the approximating solution depends critically on the ability to map spatial coordinates of an element to its parent space with high fidelity. Thus, the notion of \textit{mesh quality} - which attempts in various ways to measure the distortion of physical elements from their corresponding parent elements - is an important consideration for automated meshing tools if the resultant mesh is to be provided to a conventional FEM code. The reader is referred to the CUBIT User Documentation~\cite{cubit_2018} for a more in-depth description mesh quality metrics.

%%%%%%%%%%%%%%%%%%%%%%%%%%%%%%%%%%%%%%%%%%%%%%%
\subsection{Constitutive Update}

The constitutive update 

The forthcoming conversation will center around the kinematics, update, and subsequent contribution to the global stiffness equations for \textit{hypoelastic} materials of the form $\mathring{\bm{T}} = \bm{C}(\bm{D} - \bm{D}^p)$.

Cauchy stress

Jaumann rate

introduce incremental objectivity ...

\subsubsection{Incremental Kinematics}

NEED TO DEFINE DEF GRAD AT BEG STEP AND END STEP EARLIER WITH EVERYTHING ELSE

Much in the same way we approximate $\bm{u}$ by a linear combination of basis functions, we make the same approximations for $\bm{u}^*$ and $\hat{\bm{u}}$.

\begin{align}
u_i = u_{ia}\phi_a, \qquad u^*_i &= u^*_{ia}\phi_a, \qquad \hat{u}_i = \hat{u}_{ia}\phi_a \\
u_{i,j} = u_{ia}\phi_{a,j}, \qquad u^*_{i,j} &= u^*_{ia}\phi_{a,j}, \qquad \hat{u}_{i,j} = \hat{u}_{ia}\phi_{a,j}
\end{align}

The deformation gradients of interest are defined as:
\begin{equation}
\bf{F} = \frac{\partial \bf{x}}{\partial \bf{X}}, \qquad \bf{\overline{F}} = \frac{\partial \bf{\overline{x}}}{\partial \bf{X}}, \qquad \bf{\hat{F}} = \frac{\partial \bf{x}}{\partial \bf{\overline{x}}}
\end{equation}
Thus, $\mathbf{F}$ can be decomposed as
\begin{equation}
\bf{F} = \bf{\hat{F}}\bf{\overline{F}}
\end{equation}
It can be shown that the deformation gradients can be written in terms of the nodal displacements and shape function gradients in the following form:
\begin{align}
F_{ij} &= \delta_{ij} + \left(u^*_{ia} + \hat{u}_{ia} \right) \phi_{a,j} = \delta_{ij} + \left(u^*_{ia} + \hat{u}_{ia} \right) B_{ja} \\
\overline{F}_{ij} &= \delta_{ij} + u^*_{ia} \phi_{a,j} = \delta_{ij} + u^*_{ia} B_{ja}
\end{align}
where $B_{ja}$ are the components of the shape function gradient for node a and element $e$, evaluated at $t = t_{n+1}$, and $\delta_{ij}$ are the components of the Kronecker delta function.

MENTION CALCS FOR RESIDUAL

GO THROUGH NON INCR KINEMATICS CALCS FOR TANGENT

refer heavily to my finite defs page

If only D and W appear as kinematic forcing functions, then the only information needed to advance the material state from an initial to a final state is the values of the material state variables at the initial state, and a description of the motion – via D and W – over the time interval.

The task of the constitutive update module in a finite element code is to integrate such rate equations forward from the beginning of the solution step to the end of the step, given the D and W that correspond to the deformation increment for the step.

Next comes a crucial observation: at a single integration point, the deformation increment for the step can be conceived, for the purposes of the constitutive update, as consisting of a stretch at constant stretch rate D with no accompanying spin, followed by an impulsive rotation with no further stretch.

algorithm

consistent tangent modulus

provide example of linear hypoelastic material

\subsubsection{Hyperelastic Materials}

Hyperelastic material models describe nonlinear elastic behavior for finite deformations, and are commonly utilized to model the mechanical behavior of biological or rubber-like materials. A body comprised of a hyperelastic material that undergoes mechanical deformation is an example of a  \textit{conservative system}, and thus the stress may be written as a function of a \textit{strain energy potential} $W$. Indeed, hyperelastic materials are typically defined by the strain energy, rather than by a particular stress measure.

For hyperelastic materials, the strain energy $W$ is typically defined as a function of the deformation gradient $\bm{F}$ or a related kinematic quantity, from which the stress may be computed via its relationship with $W$. In the context of a constitutive update, in contrast to the hypoelastic case, the only kinematic quantity required to compute the Cauchy stress $\bm{T}$ is the end step deformation gradient $\bm{F}$. For the purposes of the global stiffness equations, rather than the tangent modulus ${\partial \tilde{\bm{T}}}/{\partial (\bm{D}\Delta{t})}$, the quantity ${\partial \bm{T}}/{\partial \bm{F}}$ may be computed instead. Together with $\partial \bm{F}/\partial \hat{\bm{u}}$, $\partial \bm{T}/\partial \hat{\bm{u}}$ may be computed, which is ultimately the desired quantity for those equations. In this manner, no forward rotation of the end step stress or tangent modulus is necessary for a hyperelastic material.

A hyperelastic material may still be implemented within the incremental kinematics framework previously discussed, however. This may be desirable if it is preferred that the finite element codebase treat all materials in the same manner, as opposed to performing the constitutive update and subsequent contributions to the residual equations in a fundamentally different way for hyperelastic vs. hypoelastic materials. The process requires storing the deformation gradient as a state variable, and accurately updating it within the constitutive update. The details are henceforth described for an implementation of the Mooney-Rivlin hyperelastic material model within such a framework, as a means of demonstration.

\textbf{Material Model Definition}

A compressible Mooney-Rivlin material is defined by the following relationship:
\begin{align}
W = C_1(\overline{I}_1 - 3) + C_2(\overline{I}_2 - 3) + D_1(J - 1)^2
\end{align}
where $W$ is the strain energy density, $C_1$ and $C_2$ are constants related to distortional response, and $D_1$ is a constant related to volumetric response. The quantities $\overline{I}_1 = J^{-2/3}I_1$, $\overline{I}_2 = J^{-4/3}I_2$, where $I_1$ and $I_2$ are the first and second invariants of $\bm{B} = {\bm F}{\bm F}^T$, and $J = \text{det}({\bm F})$. Specifically, $I_1 = \lambda_1^2 + \lambda_2^2 + \lambda_3^2$ and $I_2 = \lambda_1^2\lambda_2^2 + \lambda_2^2\lambda_3^2 + \lambda_1^2\lambda_3^2 = \frac{1}{2}[(\text{tr}{\bm B})^2 - \text{tr}({\bm B}^2)]$, where $\lambda_i$ are the eigenvalues of the deformation gradient ${\bm F} = {\partial {\bm x}}/{\partial {\bm X}}$.

The relationship between Cauchy stress ${\bm T}$ and strain energy density $W$ is as follows:
\begin{align}
{\bm T} = \frac{1}{J}\frac{\partial W}{\partial {\bm F}}{\bm F}^{T}
\end{align}
Using this relationship, one may obtain a direct expression for the Cauchy stress ${\bm T}$ in terms of the deformation gradient:
\begin{align}
{\bm T} = \frac{2}{J}\left[\frac{1}{J^{2/3}}(C_1 + \overline{I}_1{C_2}){\bm B} - \frac{1}{J^{4/3}}C_2{\bm B}^2\right] + \left[2D_1(J-1) - \frac{2}{3J}(C_1\overline{I}_1 + 2C_2\overline{I}_2)\right]{\bm I}
\end{align}
which can be written in index notation as:
\begin{align}
T_{ij} = \frac{2}{J}\left[\frac{1}{J^{2/3}}(C_1 + \overline{I}_{1}C_2)B_{ij} - \frac{1}{J^{4/3}}C_2B_{ik}B_{kj}\right] + \left[2D_1(J-1) - \frac{2}{3J}(C_1\overline{I}_{1} + 2C_2\overline{I}_{2})\right]\delta_{ij}
\end{align}
Using the relationships $\overline{I}_{1} = J^{-2/3}I_1$ and $\overline{I}_{2} = J^{-4/3}I_2$ and reorganizing, the relationship that will be used moving forward is as follows:
\begin{align}
\label{eq:stress}
T_{ij} = 2C_1J^{-5/3}\left[B_{ij} - \frac{1}{3}I_1\delta_{ij}\right] + 2C_2J^{-7/3}\left[I_1B_{ij} - B_{im}B_{mj} - \frac{2}{3}I_2\delta_{ij}\right] + 2D_1(J-1)\delta_{ij}
\end{align}
In the limit of small strains, this reduces to a linear elastic material if the bulk modulus $K = 2D_1$ and the shear modulus $\mu = 2(C_1 + C_2)$.

\textbf{Stress Update}

In the absence of rotation, ${\bm L} = {\bm D} = \dot{\bm F}{\bm F}^{-1} = \dot{\bm U}{\bm U}^{-1}$. We seek the appropriate $\hat{\bm U}$ that produces the same stretch that $\hat{\bm F}$ does:
\begin{align}
\dot{\bm U} &= {\bm D}{\bm U} \\
{\bm U}(t_n) &= {\bm I} \\
{\bm U}(t_{n+1}) &= \hat{\bm U}
\end{align}
The solution to this ODE is
\begin{align}
\hat{\bm U} = \text{exp}({\bm D}\Delta t)
\end{align}
We now redefine ${\bm D}$ as the stretch rate \textit{multiplied by the time step} for convenience. Thus, in terms of the original definition of ${\bm D}$ as the stretch rate, ${\bm D}\leftarrow{\bm D}\Delta t$. The previous relationship now becomes 
\begin{align}
\hat{\bm U} = \text{exp}({\bm D})
\end{align}
The Taylor expansion of this relationship around $\bm{D} = \bm{0}$ is then:
\begin{align}
\hat{\bm {U}} &= {\bm I} + {\bm D} + \frac{1}{2}{\bm D}^2 + \frac{1}{6}{\bm D}^3
\end{align}
We now define $\tilde {\bm F}$ as the deformation gradient at time $t_{n+1}$ prior to applying the rotation increment $\hat{\bm R}$. Specifically,
\begin{align}
\tilde {\bm F} = \hat{\bm U}\overline{\bm F} \\
{\bm F} = \hat{\bm R}\tilde {\bm F}
\end{align}
where $\overline{\bm F} = \partial {\overline{{\bm u}}}/\partial {\bm X}$ is the deformation gradient at time $t_n$.

Thus, when calculating stresses for the new time step, rather than feed ${\bm F}$ to calculate ${\bm T}$, $\overline {\bm F}$ is stored as a state variable, $\overline{\bm F}$ is updated via $\tilde {\bm F} = \hat{\bm U}\overline{\bm F} $, and then $\tilde{\bm{F}}$ is fed into the constitutive model to produce $\tilde{\bm T}$. Finally, outside of the constitutive update subroutine, the end step unrotated Cauchy stress and deformation gradient are forward rotated via ${\bm T} = \hat{\bm R}\tilde{\bm T}\hat{\bm R}^T$ and $\bm{F} = \hat{\bm{R}}\tilde{\bm{F}}$.

\textbf{Tangent Modulus}

If we are stepping from time $t_n$ to $t_{n+1}$, we seek the derivatives $\partial{\tilde{\bm T}}/\partial {\bm D}$, where $\tilde{\bm T}$ is the the Cauchy stress at time $t_{n+1}$ prior to applying the rotation increment $\hat{\bm R}$, and ${\bm D}$ retains its redefinition as stretch rate \textit{multiplied by the time step}.

Successive use of chain rule will be used:
\begin{align}
\frac{\partial \tilde{\bm T}}{\partial \bm D} &= \frac{\partial \tilde{\bm T}}{\partial \tilde{\bm F}}\frac{\partial \tilde{\bm F}}{\partial {\bm D}} \\
\label{eq:tanmod}
\frac{\partial \tilde{T}_{ij}}{\partial D_{kl}} &= \frac{\partial \tilde{T}_{ij}}{\partial \tilde{F}_{mn}}\frac{\partial \tilde{F}_{mn}}{\partial D_{kl}}
\end{align}

Chain rule is used on the  derivative $\partial \tilde {\bm F}/\partial {\bm D}$ as follows:
\begin{align}
\frac{\partial \tilde{\bm F}}{\partial {\bm D}} &= \frac{\partial \tilde{\bm F}}{\partial {\hat {\bm U}}}\frac{\partial \hat{\bm U}}{\partial {\bm D}} \\
\frac{\partial \tilde{F}_{ij}}{\partial D_{kl}} &= \frac{\partial \tilde{F}_{ij}}{\partial \hat{U}_{mn}}\frac{\partial \hat{U}_{mn}}{\partial D_{kl}}
\end{align}

The following tensor derivatives will be used repeatedly in what follows:
\begin{align}
\text{if } {\bm A} \neq {\bm A}^T\text{,}\ \ \ &\frac{\partial A_{ij}}{\partial A_{kl}} = \delta_{ik}{\delta_{jl}} \\
\text{if } {\bm A} = {\bm A}^T\text{,}\ \ \ &\frac{\partial A_{ij}}{\partial A_{kl}} = \frac{1}{2}(\delta_{ik}{\delta_{jl}} + \delta_{il}{\delta_{jk}})
\end{align}
Note, \textit{these definitions are different for symmetric and nonsymmetric tensors}.

The Taylor expansion definition of $\hat{\bm U}$ from the previous section is repeated here in index notation:
\begin{align}
\hat{U}_{ij} &= \delta_{ij} + D_{ij} + \frac{1}{2}D_{im}D_{mj} + \frac{1}{6}D_{im}D_{mn}D_{nj}
\end{align}
The derivative $\partial \hat{\bm U}/{\partial {\bm D}}$ is then:
\begin{equation}
\begin{split}
\frac{\partial \hat{U}_{ij}}{\partial D_{kl}} = &\frac{1}{2}\left[\delta_{ik}\delta_{jl} + \frac{1}{2}\delta_{ik}\delta_{ml}D_{mj} + \frac{1}{2}D_{im}\delta_{mk}\delta_{jl}\  + \right.\\
&\ \left.\ \ \frac{1}{6}\delta_{ik}\delta_{ml}D_{mn}D_{nj} + \frac{1}{6}D_{im}\delta_{mk}\delta_{nl}D_{nj} + \frac{1}{6}D_{im}D_{mn}\delta_{nk}\delta_{jl}\right] + \\
&\frac{1}{2}\left[\delta_{il}\delta_{jk} + \frac{1}{2}\delta_{il}\delta_{mk}D_{mj} + \frac{1}{2}D_{im}\delta_{ml}\delta_{jk}\  + \right.\\
&\ \left.\ \ \frac{1}{6}\delta_{il}\delta_{mk}D_{mn}D_{nj} + \frac{1}{6}D_{im}\delta_{ml}\delta_{nk}D_{nj} + \frac{1}{6}D_{im}D_{mn}\delta_{nl}\delta_{jk}\right]
\end{split}
\end{equation}
Further simplifying,
\begin{align}
\frac{\partial \hat{U}_{ij}}{\partial D_{kl}} = &\frac{1}{2}\left[\delta_{ik}\delta_{jl} + \frac{1}{2}\delta_{ik}D_{lj} + \frac{1}{2}D_{ik}\delta_{jl} + \frac{1}{6}\delta_{ik}D_{ln}D_{nj} + \frac{1}{6}D_{ik}D_{lj} + \frac{1}{6}D_{im}D_{mk}\delta_{jl}\right] + \\
&\frac{1}{2}\left[\delta_{il}\delta_{jk} + \frac{1}{2}\delta_{il}D_{kj} + \frac{1}{2}D_{il}\delta_{jk} + \frac{1}{6}\delta_{il}D_{kn}D_{nj} + \frac{1}{6}D_{il}D_{kj} + \frac{1}{6}D_{im}D_{ml}\delta_{jk}\right]
\end{align}

Now, repeating the definition for $\tilde {\bm F}$ in index notation,
\begin{align}
\tilde{F}_{ij} = \hat{U}_{im}\overline{F}_{mj}
\end{align}
The derivative ${\partial \tilde{\bm F}}/{\partial \hat{\bm U}}$ is
\begin{align}
\frac{\partial \tilde{F}_{ij}}{\partial \hat{U}_{kl}} = \frac{1}{2}\left(\delta_{ik}\delta_{ml}\overline{F}_{mj} + \delta_{il}\delta_{mk}\overline{F}_{mj}\right)
\end{align}
and simplifying:
\begin{align}
\frac{\partial \tilde{F}_{ij}}{\partial \hat{U}_{kl}} &= \frac{1}{2}\left(\delta_{ik}\overline{F}_{lj} + \delta_{il}\overline{F}_{kj}\right)
\end{align}

With $\partial \tilde{\bm F}/\partial \hat{\bm U}$ and $\partial \hat{\bm U}/\partial {\bm D}$ specified, $\partial \tilde{\bm F}/\partial {\bm D}$ is fully defined. What remains to fully define the tangent modulus $\partial \tilde{\bm T}/\partial {\bm D}$ is to calculate $\partial \tilde{\bm T}/\partial \tilde {\bm F}$. We may write the derivative as follows:
\begin{align}
\frac{\partial \tilde{\bm T}}{\partial \tilde{\bm F}} &= \frac{\partial \tilde{\bm T}}{\partial J}\frac{\partial J}{\partial \tilde {\bm F}} + \frac{\partial \tilde{\bm T}}{\partial {\tilde{\bm {B}}}}\frac{\partial {\tilde{\bm {B}}}}{\partial \tilde {\bm F}} \\
\frac{\partial \hat{T_{ij}}}{\partial \tilde{F}_{kl}} &= \frac{\partial \tilde{T}_{ij}}{\partial J}\frac{\partial J}{\partial \tilde{F}_{kl}} + \frac{\partial \tilde{T}_{ij}}{\partial \tilde{B}_{mn}}\frac{\partial \tilde{B}_{mn}}{\partial \tilde{F}_{kl}}
\end{align}
where $J = \det\tilde{\bm{F}}$ and $\tilde{\bm{B}} = \tilde{\bm{F}}\tilde{\bm{F}}^T$.

The corresponding derivatives are then:
\begin{align}
\frac{\partial J}{\partial {\tilde{F}}_{kl}} &= {J}{\tilde{F}}^{-1}_{lk} \\
\frac{\partial \tilde{B}_{ij}}{\partial {\tilde{F}}_{kl}} &= \delta_{ik}\delta_{ml}{\tilde{F}}_{jm} + {\tilde{F}}_{im}\delta_{jk}\delta_{ml} \\
 &=  \delta_{ik}{\tilde{F}}_{jl} + {\tilde{F}}_{il}\delta_{jk}
\end{align}
The relationship for the updated Cauchy stress is repeated here:
\begin{equation}
\begin{aligned}
T_{ij} = &\ 2C_1J^{-5/3}\left[\tilde{B}_{ij} - \frac{1}{3}I_1\delta_{ij}\right] + \\ 
&\ 2C_2J^{-7/3}\left[I_1\tilde{B}_{ij} - \tilde{B}_{im}\tilde{B}_{mj} - \frac{2}{3}I_2\delta_{ij}\right] + \\
&\ 2D_1(J-1)\delta_{ij}
\end{aligned}
\end{equation}
Then,
\begin{equation}
\begin{aligned}
\frac{\partial \tilde{T}_{ij}}{\partial J} = &\frac{-10}{3}C_1J^{-8/3}\left[\tilde{B}_{ij} - \frac{1}{3}I_1\delta_{ij}\right] + \\
&\ \frac{-14}{3}C_2J^{-10/3}\left[I_1\tilde{B}_{ij} - \tilde{B}_{im}\tilde{B}_{mj} - \frac{2}{3}I_2\delta_{ij}\right] + \\ &\ 2D_1\delta_{ij}
\end{aligned}
\end{equation}

To calculate the derivative $\partial \tilde{\bm T}/\partial {\tilde{\bm {B}}}$, the following relationships will be needed:
\begin{align}
\frac{\partial I_1}{\partial \tilde{B}_{kl}} &= \delta_{lk}  \\
\frac{\partial I_2}{\partial \tilde{B}_{kl}} &= I_1\delta_{kl} - \tilde{B}_{lk}
\end{align}
Finally,
\begin{equation}
\begin{aligned}
\frac{\partial \tilde{T}_{ij}}{\partial \tilde{B}_{kl}} = &\ 2C_1J^{-5/3}\left[\frac{1}{2}\left(\delta_{ik}\delta_{jl} + \delta_{il}\delta_{jk}\right) - \frac{1}{3}\delta_{ij}\delta_{kl}\right] + \\
&\ 2C_2J^{-7/3}\left[\delta_{kl}\tilde{B}_{ij} + \frac{1}{2}I_1\left(\delta_{ik}\delta_{jl} + \delta_{il}\delta_{jk}\right) -\frac{1}{2}\left(\delta_{ik}\delta_{ml} + \delta_{ik}\delta_{ml}\right)\tilde{B}_{mj} \right. + \\
&\phantom{xxxxxxxx}-\frac{1}{2}\tilde{B}_{im}\left(\delta_{mk}\delta_{jl} +\delta_{ml}\delta_{jk}\right) 
\left.- \frac{2}{3}I_1\delta_{ij}\delta_{kl} + \frac{2}{3}\delta_{ij}\tilde{B}_{lk}\right]
\end{aligned}
\end{equation}
And simplifying,
\begin{equation}
\begin{aligned}
\frac{\partial \tilde{T}_{ij}}{\partial \tilde{B}_{kl}} = &\ 2C_1J^{-5/3}\left[\frac{1}{2}\delta_{ik}\delta_{jl} + \frac{1}{2}\delta_{il}\delta_{jk} - \frac{1}{3}\delta_{ij}\delta_{kl}\right] + \\
&\ 2C_2J^{-7/3}\left[\delta_{kl}\tilde{B}_{ij} + \frac{1}{2}I_1\delta_{ik}\delta_{jl} + \frac{1}{2}I_1\delta_{il}\delta_{jk} -\frac{1}{2}\delta_{ik}\tilde{B}_{lj} -\frac{1}{2}\delta_{il}\tilde{B}_{kj} \right. + \\
&\left.\phantom{xxxxxxxx}-\frac{1}{2}\tilde{B}_{ik}\delta_{jl} -\frac{1}{2}\tilde{B}_{il}\delta_{jk} - \frac{2}{3}I_1\delta_{ij}\delta_{kl} + \frac{2}{3}\delta_{ij}\tilde{B}_{lk}\right]
\end{aligned}
\end{equation}
Thus, all terms have been defined in the calculation of the tangent modulus $\partial{\tilde{\bm{T}}}/\partial({\bm{D}\Delta{t}})$. Like the stress and state variables, the tangent modulus is forward-rotated elsewhere in the finite element code prior to contributing to the global stiffness equations, in the same manner described previously.

The procedure for implementing a hyperelastic material into an FEM code that only accepts hypoelastic materials has thus been established, and will be used again in \chapref{5} for implementing the mechanical behavior of cardiac tissue.

\subsubsection{Incompressibility}

Special measures must be taken to avoid \textit{volumetric locking} for materials with an incompressibility constraint. Volumetric locking occurs when there is a large mismatch between the material's apparent stiffness in deviatoric deformation vs. volumetric deformation~\cite{rashid_2017}. The most common approach to \textit{fully} enforcing incompressibility is a \textit{mixed pressure-displacement} formulation, in which the pressure is treated as an independent variable and interpolated independently from displacement. For displacement-only formulations, the constraint may be enforced in a \textit{nearly incompressible} sense.

Enforcement of near-incompressibility on an integration-point level results in the displacement solution being too small, i.e., the mesh becomes artificially too stiff. This volumetric locking phenomenon occurs because modes of deformation (refered to as \textit{hourglass modes} for quad or hex elements) are prevented even though they preserve the volume on an element level.

Thus, the constraint is relaxed by enforcing incompressibility only on an element level. This amounts to modifying the incremental deformation gradient $\hat{\bm{F}}$ at each of the integration points of an element such that the dilatation is replaced by an element-averaged value, while the deviatoric deformation is unmodified. Thus, at each of the integration points, $\hat{\bm{F}}$ is replaced by $\doublehat{\bm{F}}$ in the following manner:
\begin{align}
\doublehat{\bm{F}}  &= \left(\frac{A}{\hat{J}}\right)^{1/3}\hat{\bm{F}} \\
A &= \det\bm{G} \\
\bm{G} &= \frac{1}{|\overline{\Omega}|} \int_{\overline{\Omega}} \hat{\bm{F}}dv
\end{align}
where $\hat{J} = \det\hat{\bm{F}}$, $\overline{\Omega}$ is the element at its beginning-step configuration, and ${|\overline{\Omega}|}$ is its volume.

It should be noted that $\det\doublehat{\bm{F}}$ only approximates the overall volume change of the corresponding element. Nonetheless, the approach has been shown to approximate the exact solution under mesh refinement with sufficient accuracy for hex-8 and quad-4 elements. Refer to Rashid~\cite{rashid_2017} and Doll \textit{et al.}~\cite{doll_2000} for a more detailed description of volumetric locking for incompressible materials, how to address it, and a more detailed commentary on the F-bar method.

\subsection{Boundary Conditions}
Displacement boundary conditions on $\partial_u\kappa_0$ are enforced on the corresponding nodal displacements belonging to that surface. In the event that the prescribed displacements do not depend on the solution, the corresponding nodal values are no longer considered unknowns. The size of the nonlinear system of residual equations is subsequently reduced to the DOFs of the nodes belonging to the interior of the body and those belonging to $\partial_t\kappa_0$.

Traction boundary conditions prescribe the Piola traction $\overline{\bm{p}}$ in the residual equations over the patch of facets comprising $\partial_t\kappa_0$. The Piola traction may be directly specified, or alternatively the Piola pressure, Cauchy traction, or Cauchy pressure may be enforced. In the case of the latter three options, the boundary condition must be converted to a Piola traction to fit into the form of \eqnref{}. Enforcing the latter three BCs in a finite deformation setting is nontrivial. Specifically, facet contributions to the residual and global stiffness equations depend on the area ratio $\alpha$ and current configuration normal $\bm{n}$ to be computed, along with their derivatives with respect to incremental nodal displacements.

Cauchy and Piola \textit{follower loads} may also be considered, in which case the normal and shear tractions ``follow'' the local orientation of the facet in the current configuration. Those BCs are less common and will not be discussed here, however. The remaining discussion revolves around the implementation of Piola tractions, Piola pressures, Cauchy tractions, and Cauchy pressures into a nonlinear finite element code that stores \textit{facet elements} for the enforcement of boundary conditions. The task involves computing $\overline{\bm{p}}$ for the residual and ${\partial \bm{\overline{p}}}/{\partial {\hat {\bm{u}}}}$ for the tangent stiffness.

Calculations for the traction boundary conditions all stem from the formation of ${\bm {da}} = \alpha{\bm n}$ in the current configuration. The quantity $\alpha{\bm n}$ is found via the cross product of orthonormal in-plane deformed material line directions in the current configuration: ${\alpha}{\bm n} = {\bm F}{\bm M}_1 \times {\bm F}{\bm M}_2$. For each integration point of a particular facet element, the four boundary conditions mentioned are then enforced based on the following calculations:

If Cauchy pressure $\overline{p}_c$ is prescribed, then $\bm {\overline{p}} = {-\overline{p}_c}\alpha{\bm n}$ and $\frac{\partial \bm {\overline{p}}}{\partial \hat {\bm{u}}} = {-\overline{p}_c}\frac{\partial (\alpha {\bm n})}{\partial {\hat {\bm{u}}}}$, where:
\begin{align}
\alpha{\bm n} &= {\bm F}{\bm M}_1 \times {\bm F}{\bm M}_2 \\
\frac{\partial (\alpha{\bm n})}{\partial {\hat {\bm{u}}}} &= (\frac{\partial {\bm F}}{\partial {\hat {\bm{u}}}}{\bm M}_1 \times {\bm F}{\bm M}_2) + ({\bm F}{\bm M}_1 \times \frac{\partial {\bm F}}{\partial {\hat {\bm{u}}}}{\bm M}_2)
\end{align}
If Cauchy traction $\overline{\bm{t}}$ is prescribed, then $\bm {\overline{p}} = \alpha{\overline{\bm{t}}}$ and $\frac{\partial \bm {\overline{p}}}{\partial {\hat {\bm{u}}}} = \frac{\partial \alpha}{\partial {\hat {\bm{u}}}}{\overline{\bm{t}}}$, where:
\begin{align}
{\alpha \bm n}, \ \frac{\partial (\alpha {\bm n})}{\partial {\hat {\bm{u}}}} &\text{ computed as before} \nonumber \\
\alpha &= [\alpha {\bm n} \bm{\cdot} \alpha {\bm n}]^{1/2} \\
 \frac{\partial{\alpha}}{\partial {\hat {\bm{u}}}} &= \frac{1}{\alpha}\frac{\partial({\alpha {\bm n}})}{\partial {\hat {\bm{u}}}} \bm{\cdot} ({\alpha \bm n})
\end{align}
If Piola pressure $\overline{p}_p$ is prescribed, then $\bm{\overline{p}} = {-\overline{p}_p}{\bm n}$ and $\frac{\partial \bm {\overline{p}}}{\partial {\hat {\bm{u}}}} = {-\overline{p}_p}\frac{\partial {\bm n}}{\partial {\hat {\bm{u}}}}$, where:
\begin{align}
{\alpha \bm n}, \ &\frac{\partial (\alpha {\bm n})}{\partial {\hat {\bm{u}}}}, \ \alpha, \ \frac{\partial \alpha}{\partial {\hat {\bm{u}}}} \text{ computed as before} \nonumber \\
{\bm n} &= \frac{1}{\alpha}(\alpha {\bm n}) \\
\frac{\partial {\bm n}}{\partial {\hat {\bm{u}}}} &= -\frac{1}{\alpha^2}(\alpha {\bm n}) + \frac{1}{\alpha}\frac{\partial (\alpha{\bm n})}{\partial {\hat {\bm{u}}}}
\end{align}

Finally, If Piola traction $\bm {\overline{p}}$ is prescribed: $\bm {\overline{p}}$ is explicitly defined and $\frac{\partial \bm {\overline{p}}}{\partial {\hat {\bm{u}}}} = {\bm 0}$.

In order to perform these steps, the deformation gradient $\bm{F}$ and its derivative ${\partial \bm{F}}/{\partial \hat{\bm{u}}}$, and local material line directions $\bm{M}_1$ and $\bm{M}_2$ in the current configuration must be computed for each integration point of each facet element. The remaining discussion is centered around computing those values.

For three-dimensional finite elements, the element residual and tangent stiffness for facets involve area integrals in $\mathbb{R}^3$ that depend on $\alpha$, $\bm{n}$, and their derivatives. In order to compute these quantities for a particular facet, the following are required: 1) shape function values $N_a$, 2) the ratio $J_F$ of differential areas between reference configuration and parent space, 3) the reference configuration normal ${\bm {N}}$, and 4) the in-plane shape function gradients ${\bm {g}}_a$. Shape function values $N_a$ can be easily evaluated at the integration points. 

To determine these quantities, assume a parametrization $\mathbf{X} \in \Omega^m$ for a particular element $m$ is available in the reference configuration in the physical space. For conventional finite element methods, the parameterization is provided by shape functions defined in the parent space: ${X}_{i} = \sum\limits_{a}{X}_{ia}{N}_a(\xi_i)$, where parameterization variables $\xi_i$ are simply the parent space coordinates in conventional FEM. The cross product of ${\partial{\bm {X}}}/{\partial \xi_1}$ and ${\partial{\bm {X}}}/{\partial \xi_2}$ defines a vector that points normal to the facet in the reference configuration, with magnitude equal to the differential area spanned by ${\partial{\bm {X}}}/{\partial \xi_1}$ and ${\partial{\bm {X}}}/{\partial \xi_2}$. This magnitude and direction are indeed the $J_F$ and ${\bm {N}}$ we desire, respectively.

With $\bm{N}$ defined, the local orthonormal in-plane directions in the current configuration $\bm{M}_1$ and $\bm{M}_2$ straightforwardly follow. Calculation of the material line segment ${\bm {M}}_1$ is performed by choosing the basis vector ${\bm {e}}_p$ that is closest to lying in the plane defined by ${\bm N}$, projecting it onto the plane defined by $\bm{N}$, and scaling it to be unit magnitude. The second direction ${\bm M}_2$ is calculated simply via ${\bm {N}} \times {\bm {M}}_1$.

In-plane shape function gradients $\bm{g}_a$ remain, which are used to reconstruct the deformation gradient $\bm{F}$ and its derivatives within the facet integral subroutine. We seek to calculate the in-plane shape function gradients in the reference configuration $\bm{g}_a = \frac{\partial N_a}{\partial \bm{X}}$ in terms of $\bm{M}_1$ and $\bm{M}_2$. We may write $\bm{g}_a$ as follows:
\begin{equation}
{\bm {g}}_a = {h_{1a}}{\bm {M}}_1 + {h_{2a}}{\bm {M}}_2
\end{equation}
where the coefficients $h_{1a}$ and $h_{2a}$ must then be calculated.

If we define ${\bm {M}}$ as the 3$\times$2 matrix whose columns are ${\bm {M}}_1$ and ${\bm M}_2$, and if $\mathbf{h}_a$ is the 2$\times$1 column vector with components $h_{1a}$ and $h_{2a}$, we may write ${\bm g}_a$ as ${\bm g}_a = {\bm M}{\bm h}_a$. We may then define shape function gradients with respect to the parameterization variables as $\frac{\partial N_a}{\bm{\partial {\xi}}} = {\bm g}_a \bm{\cdot} \frac{\partial {\bm X}}{\bm{\partial {\xi}}}$. Using the definition for ${\bm g}_a$ above, this can be written as:
\begin{align}
\frac{\partial N_a}{\partial {\bm \xi}} &= {\bm g}_a \bm{\cdot} \frac{\partial {\bm X}}{\partial {\bm \xi}} \\
&= {\bm M}{\bm h}_a \bm{\cdot} \frac{\partial {\bm X}}{\partial {\bm \xi}} \\
&= (\frac{\partial {\bm X}}{\partial {\bm \xi}}^T{\bm M}){\bm h}_a \\
&= {\bm H}{\bm h}_a
\end{align}
The matrix ${\bm H} = \frac{\partial {\bm X}}{\partial {\bm \xi}}^T{\bm M}$ maps in-plane shape function gradients $h_{ia}$ acting in directions ${\bm M}_i$ into shape function gradients in directions of the parameterized variables ${\xi_i}$. The components $h_{ia}$ can then be found from ${\bm h}_a = {\bm H}^{-1}\frac{\partial N_a}{\partial {\bm \xi}}$. Finally ${\bm g}_a$ is formed using the relation ${\bm g}_a = {h_{1a}}{\bm M}_1 + {h_{2a}}{\bm M}_2$.

The previous discussion simplifies to a concise set of calculations that are performed at each integration point of a facet:
\begin{align}
\frac{\partial X_{i}}{\partial \xi_j} &= \sum_a X_{i}\frac{\partial N_a}{\partial \xi_j}, \text{\ \ \ \ }i=1,3, \text{\ \ }j = 1,2 \\
{\bm{dA}} &= \frac{\partial{\bm{X}}}{\partial \xi_1} \times \frac{\partial {\bm {X}}}{\partial \xi_2} \\
J_F &= | {\bm {dA}}| \\
{\bm N} &= \frac{1}{J_F}{\bm {dA}} \\
\text{Determine } p &\text{ such that } N_p = \min_i{|N_i|} \\
{\bm M}_1 &=\frac{ ({\bm I} - {\bm N}\otimes{\bm N}){\bm e}_p}{|({\bm I} - {\bm N}\otimes{\bm N}){\bm e}_p|} \\
{\bm M}_2 &= {\bm N} \times {\bm M}_1 \\
H_{ij} &= \frac{\partial {\bm X}}{\partial \xi_i} \bm{\cdot} {\bm M}_j, \text{\ \ \ \ }i=1,\text{2}, \text{\ \ }j = 1,\text{2} \\
{\bm h}_a &= {\bm H}^{-1} \frac{\partial N_a}{\partial {\bm \xi}} \\
{\bm g}_a &= h_{1a}{\bm M}_1 + h_{2a}{\bm M}_2 \\
F_{ij} &= \delta_{ij} + \sum_a{u_{ia}g_{ja}} \\
\frac{\partial F_{ij}}{\partial \hat{u}_{kb}} &= \delta_{ik}g_{jb}
\end{align}

Thus, $\bm{F}$, ${\partial \bm{F}}/{\partial \hat{\bm{u}}}$, $\bm{M}_1$, and $\bm{M}_2$ have all been defined and can be used to compute the various boundary conditions described previously.

\subsection{Additional Topics}
The finite element method is a rich field with many more pertinent topics that play an important role in producing accurate and efficient approximations to physical phenomena. Additional important topics, to name only a few, include: additional nonlinear solutions schemes to the residual equations; direct and implicit solvers for large matrix computations; mesh refinement/convergence studies and \textit{a posteriori} error estimation; 
enforcement of fully incompressible materials, for example with a mixed pressure-displacement formulation; numerical modeling of frictionless and frictional contact of multiple bodies; numerical modeling of material separation, including fracture; numerical modeling of thin structures, and additional approaches to mitigating element locking, both of the volumetric kind and otherwise; implicit and explicit approaches to modeling the dynamics of solid bodies; and coupling of multiple physical phenomena. 

%%%%%%%%%%%%%%%%%%%%%%%%%%%%%%%%%%%%%%%%%%%%%%%
%%%%%%%%%%%%%%%%%%%%%%%%%%%%%%%%%%%%%%%%%%%%%%%
\section[A Polyhedral Finite Element Method in Computational Solid \\ Mechanics]{\texorpdfstring{A Polyhedral Finite Element Method in \\ Computational Solid Mechanics}{A Polyhedral Finite Element Method in Computational Solid \\ Mechanics}}
\label{A Polyhedral Finite Element Method in Computational Solid Mechanics}

REFERENCES:
\cite{sukumar_2006}
\cite{rashid_2006, rashid_2012, rashid_2015}

look up PEM for the gifted notes

refer to other people's dissertations

Polyhedral finite element methods are an emerging class of finite-element-like methods within the general class of Galerkin approximation schemes, in which the individual elements need not conform to the topology of a fixed canonical (or “parent”) element. Polyhedral FEMs exhibit much of the same desirable properties of conventional finite elements, namely: they enjoy high quadrature efficiency, they interpolate the nodal data so that essential BCs are easy to enforce, and they support convenient equation assembly and modular code architecture. The specific implementation of polyhedral FEM in Celeris is known as the “Partitioned Element Method”. The Partitioned Element Method employs a geometric partition of each polyhedral element into quadrature cells. The shape functions are then formulated discretely, on the complex of quadrature cells, via an element-local quadratic optimization problem. Arbitrary polyhedral element shapes (including non-convex) are accommodated without deterioration in local solution accuracy, making the mesh in Figure 9(c) a perfectly adequate discretization for this method. Higher-order shape functions are possible with this approach, although for this demonstration only elements with first order interpolation will be used.

To that end, the simulation resolution need only be refined to the extent that accurate solutions can be provided, rather than being constrained in some way by the number of points and polygons used to the describe the surface of the object.

%%%%%%%%%%%%%%%%%%%%%%%%%%%%%%%%%%%%%%%%%%%%%%%
