\chapter{Overview}
\textit{Image-based modeling and simulation}, also known as \textit{patient-specific modeling and simulation} or \textit{in-silico modeling and simulation}, is the process of performing mathematical computations based on imaging data to analyze and predict the physical behavior of biological tissues. Imaging data directly provides the geometry and/or material properties of the biological structures of interest, in contrast to conventional engineering disciplines where the geometry and materials of an object originate from man-made designs. Historically, studies of biological tissues or organisms have been limited to either \textit{in vivo} (within the living) experimentation or \textit{ex vivo} (outside of the living) experimentation. \textit{In vivo} studies raise concerns about cost, study controls, and the ethics of experimenting on living beings, though, and \textit{ex vivo} studies can never truly represent how an organism behaves after it has been removed from its native environment. \textit{In silico} (in silicon) studies provide the ability to model \textit{in vivo} behavior without the same concerns of cost or ethics, and with more control of desired detail than what \textit{ex vivo} affords~\cite{colquitt_2011}. That control allows \textit{in silico} approaches to be used in conjunction with \textit{in vivo} and \textit{ex vivo} ones to advance the medical sciences in a number of exciting ways.

%%%%%

Image-based modeling and simulation has played an important predictive and analytic role in several areas, including:  medical device design optimization,  surgical planning, improved diagnosis and treatment, understanding causes of pathology, computer-assisted surgery, and surgeon training. These technologies have been applied to nearly every area of the body, including: the cardiovascular system~\cite{min_2015, updegrove_2016}, brain~\cite{weickenmeier_2016, behnia_2008}, knee~\cite{erdemir_2015, donahue_2002}, hip~\cite{anderson_2008, el'sheikh_2003}, spine~\cite{malandrino_2014, dumas_2005}, liver~\cite{shi_2008, schwen_2014}, kidney~\cite{eloot_2002, snedeker_2005}, jaw~\cite{idrus_2017, narra_2014}, and teeth~\cite{frisardi_2011, geng_2001}. 

Regulatory organizations and professional societies have identified the incredible potential of computational modeling in the medical device field. The American Society of Mechanical Engineers (ASME) and the U.S. Food and Drug Administration (FDA) formed a subcomittee in 2010 named \textit{V\&V 40: Verification and Validation in Computational Modeling of Medical Devices}, out of which a regulatory standard is expected to be published in early 2018~\cite{committee}. The Center for Devices and Radiological Health (CDRH) at the FDA identified Computational Modeling as a priority in FY2017~\cite{Morrison2017}. The FDA has published several guidelines in last few years: \textit{Reporting of Computational Modeling Studies in Medical Device Submissions}~\cite{fda1_2016}, \textit{Software as a Medical Device (SAMD): Clinical Evaluation}~\cite{fda1_2016}, and \textit{Applicability Analysis of Validation Evidence for Biomedical Computational Models}~\cite{pathmanathan_2017}. The development of tools and regulations related to \textit{software as a medical device} demonstrates the extent to which industry and regulatory bodies have invested in computational modeling to move health care forward. And, \textit{in silico} modeling and simulation lies at the heart of those development efforts.

%%%%%

The field of image-based modeling and simulation covers a broad spectrum of topics, including image processing, computational geometry, numerical methods, and biomechanics. The workflow entails: medical imaging and image segmentation, image-based mesh generation, physics-based modeling and simulation, and use in medical applications. Each of these steps are described in turn. The work described herein outlines novel contributions to certain elements of this workflow, as well as novel combinations of existing technologies.

\chapref{2} describes the basic physics involved in acquiring medical images for a number of imaging modalities. Medical images will be assumed here to be three-dimensional rectilinear grids of \textit{voxels}, or 3D pixels. The chapter also presents a review of current image segmentation techniques, which delimit the voxels in an image into the various objects and tissues of interest. \chapref{3} discusses image-based mesh generation, which involves generating geometrical meshes suitable for simulation based on segmented images. Image-based meshing is often separated into two steps: 1) surface generation,  and 2) conventional CAD-based volume meshing. A novel Voronoi-based surface generation technique is presented, and a brief review of CAD-based meshing techniques are covered. A novel image-based meshing tool \textit{Shabaka} is introduced, that makes important improvements to the speed and robustness of generating meshes from image data.

The next step in the workflow - physics-based modeling and simulation - may refer to a number of different physical phenomena and numerical methods. \chapref{4} presents pertinent theory and implementation specific to modeling nonlinear solid mechanics using finite element methods (FEM). Principles of continuum mechanics and finite element methods are discussed, as well as a particular flavor of polyhedral finite elements. An implementation of incremental kinematics within finite element codes is presented, and the use of hyperelastic materials in an incremental kinematics framework is demonstrated. \chapref{5} covers the use of image-based modeling and simulation in the field of cardiac mechanics. The important numerical considerations in performing human heart modeling are discussed. Contributions made to an existing cardiac mechanics code are highlighted, and simulation results using that code are shown.

Future work is discussed on several fronts in \chapref{6}. Progress toward demonstrating polyhedral finite element methods as a viable option in human heart modeling is presented. Potential improvements to both the image-based meshing and cardiac mechanics codes are noted. Broader topics in the future of image-based modeling and simulation are also mentioned, specifically how it is playing a role in the \textit{simulation of clinical trials} and in applications in \textit{rapid prototyping} and \textit{additive manufacturing}. A continuing project is discussed that uses machine learning in conjunction with the novel image-based meshing tool introduced in \chapref{3}. Lastly, concluding remarks are given in \chapref{7}.

For the electronic version of this document, images may be zoomed into for detail.
