\chapter{Conclusions}
\label{chap:7}
%

This work has presented a new image-based modeling and simulation workflow for use in modeling mechanical phenomena of biological tissues. The pipeline involves: medical image acquisition and image segmentation; surface and volume mesh generation; finite element methods for solving nonlinear solid mechanics problems; and finally, simulation of biomechanics applications. New and existing tools were combined to produce a robust workflow that was showcased in the field of cardiac mechanics.

The first steps in the pipeline involve image acquisition and image segmentation. Medical images are typically three-dimensional rectilinear grids of voxel data, most commonly acquired from MRI or CT techniques. These images are processed to remove noise and emphasize tissue contrast. Image segmentation techniques involve identifying the regions to which each voxel in an image belongs. The most effective approaches typically involve a combination of thresholding, region-growing, and edge-based, and manual techniques. Third-party software was used to read, process, and segment images.

The surface generation step features a novel algorithm for generating b-reps from binary image masks. The algorithm first generates a dense oriented point cloud that approximates the surface of the object by minimizing a relative error functional. This error measures the difference in geometric properties between the voxelized image mask and the approximated surface. The point cloud generation task is highly parallelizable since the calculation of each point is independent of one another. A Voronoi-based surface reconstruction algorithm was presented, that inserts Voronoi sites based on the point cloud, computes a Voronoi partition, and extracts the surface based on material assignments of the Voronoi sites. The surface generation algorithm performs well for surfaces that do not have regions of excessively high curvature or sharp features. The surface is subsequently cleaned and decimated to produce a watertight, manifold surface mesh that may be used as input to CAD-based meshing software, 3D printers, or visualization programs. The code was linked with third-party image segmentation and CAD-based meshing software and packaged into a free, open-source tool named Shabaka. It is available on GitHub for Linux, Mac, and Windows. The tool is easy to install and use, and it generates meshes in matter of minutes on a personal computer. Dozens of example meshes were created to demonstrate the code's capability.

Next, third-party tetrahedral and polyhedral mesh generation tools were utilized to produce volume meshes for the purposes of physics-based simulation. For conventional FEM, robust automated hex meshing tools do not yet exist, and thus higher order tetrahedra are the most commonly used elements for general biomechanics applications. Polyhedral meshes are generated by sculpting the input b-rep by a background hex mesh in a tolerance-aware manner. The resulting mesh exhibits cuboidal polyhedral elements on the interior of the object and arbitrary polyhedra near the surface. Because conventional FEM approaches have strict geometric and topological restrictions on the elements in a mesh, the polyhedral mesh is used in conjunction with a FEM-like method that accommodates general polyhedral shapes.

Volume meshes are used as input to finite element codes for modeling nonlinear solid mechanics of three-dimensional bodies. The basic theory underlying nonlinear solid mechanics and finite element methods were reviewed. Specifics related to the implementation of various material models and boundary conditions were outlined. Polyhedral finite elements were also summarized, which offer the promise of a robust and automated workflow while potentially significantly reducing the system size compared to conventional FEM. That reduction is mainly achieved because the simulation resolution is independent of the geometric resolution of the mesh.

Finally, the workflow was demonstrated by performing image-based cardiac mechanics simulations. The surface mesh of a human heart was generated from MRI data utilizing the tools described. It was used as input to automatically produce tetrahedral and polyhedral meshes for the purposes of finite element analysis. Simulations were performed using a massively parallel code at Lawrence Livermore National Laboratory to model the mechanical behavior of a beating heart. Conventional finite element techniques were applied to a quadratic tet mesh. In the process, existing tools to model muscle fiber orientations, simulate the electrophysiology, and assign boundary conditions were streamlined. Four full beats of the deforming human heart were simulated.

Initial work was performed to consider polyhedral FEM as an alternative to conventional approaches in modeling cardiac mechanics and biomechanics problems in general. The fiber orientations, activation times, boundary condition tags, and pressure boundary conditions were imported into a commercial polyhedral FEM code for the purposes of running the same simulation using polyhedral FEM. The material model incorporated in the cardiac simulations was implemented in the polyhedral code and verified against a suite of test problems. Additional steps will include completing the verification of the polyhedral implementation, demonstrating capability of polyhedral FEM in cardiac mechanics, and showcasing the ability to produce accurate results with fewer degrees of freedom.

Image-based modeling and simulation is beginning to impact the medical field in a number of meaningful ways. For example, the workflow that has been described is being used in a project that combines simulation with machine learning techniques to improve procedures that treat cardiac arrhythmia. As patent-specific material parameters and boundary conditions become better understood, and models begin to incorporate more rigorous validation and uncertainty quantification techniques, \textit{in silico} modeling will continue to play a larger role in healthcare in the future. Simulation of clinical trials, optimization of surgical techniques and medical device designs, and additive manufacturing are among the most compelling and impactful areas that are benefiting from this technology.