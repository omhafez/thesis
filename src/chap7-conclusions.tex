\chapter{Conclusions}
\label{chap:7}
%

meshing:
heuristics still play a major role, arguably equally as important as the theory
neural nets may well play an integral role in the future


Polyhedral finite element methods (PFEM), together with polyhedral meshing tools, can offer a more robust and automated meshing pipeline for simulations involving complex geometries, and can significantly reduce the system size compared to conventional finite element approaches to cardiac mechanics. The system size reduction is accomplished first because the polyhedral meshes produced are linear hex-dominant, and second because the simulation mesh resolution can be generated independently of the resolution of an input surface mesh.


Overall, the work described herein produced some novel image-based meshing techniques; identified critical bugs in the polyhedral, fiber generation, and ODE code generation tools; significantly improved Cardioid’s image-based modeling and simulation workflow, from which the current machine learning project will benefit immediately; and has shown promising preliminary results in demonstrating PFEM as an alternative to conventional approaches in modeling cardiac mechanics.