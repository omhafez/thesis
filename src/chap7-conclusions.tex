\chapter{Conclusions}
\label{chap:7}
%

This work presented a new image-based modeling and simulation pipeline. The workflow includes a novel algorithm for generating b-reps from binary image masks. The algorithm first generates a dense oriented point cloud approximating the surface of the object by minimizing a relative error functional measuring the difference in geometric properties between the voxelated image mask and the approximated surface. The point cloud generation step is highly parallelizable since the calculation of each point is independent of one another. A Voronoi-based surface reconstruction algorithm was presented, that inserts Voronoi sites based on the point cloud, computes a Voronoi partition, and extracts the surface based on material assignments of the Voronoi sites. The surface generation algorithm performs well for surfaces that do not exhibit regions of excessively high curvature or sharp features. The surface is subsequently cleaned and decimated to produce a watertight, manifold surface mesh that may be used as input to CAD-based meshing software, 3D printers, or visualization programs. The code was linked with third-party image segmentation and CAD-based meshing software, and packaged into a free, open-source tool named Shabaka, available on Linux, Mac, or Windows. The tool installs in a matter of 15 minutes, and meshes are generated from image masks typically in less than 5 minutes on a personal computer. Dozens of example meshes were created to demonstrate capability.

The surface mesh of a human heart was generated using the tools described, and used as input to produce quadratic tetrahedral and polyhedral bi-ventricular meshes for the purposes of finite element analysis. Cardiac mechanics simulations were performed on massively parallel systems to model the mechanical behavior of a beating human heart using quadratic tets within the code Cardioid. In the process, existing tools to model muscle fiber orientations, simulate the electrophysiology, and assign boundary conditions were streamlined for expedient results. The fiber orientations, boundary condition tags, activation times, and pressure boundary conditions were imported into the polyhedral code Celeris for the purposes of running the same simulation using polyhedral FEM. Polyhedral finite elements offer the promise of a robust and automated image-based modeling and simulation workflow while potentially significantly reducing the system size compared to conventional FEM approaches. That reduction is most importantly achieved because the simulation mesh resolution is completely independent of the geometric resolution. The material model involved in producing the cardiac simulations was implemented in the polyhedral code and verified against a suite of test problems. Additional steps will complete the verification of the polyhedral implementation, demonstrate capability of polyhedral FEM in cardiac mechanics, and showcase the ability to produce accurate results with fewer degrees of freedom.The work has shown promising preliminary results in demonstrating PFEM as an alternative to conventional approaches in modeling cardiac mechanics.

The image-based modeling and simulation workflow is to be used in a project that performs cardiac simulations in conjunction with machine learning techniques to improve procedures treating cardiac arrhythmia. Provided that the role of patent-specific material parameters are patient-specific boundary conditions are better understood, and models begin to incorporate more rigorous validation and uncertainty quantification techniques, \textit{in silico} modeling has the potential to significantly impact how medicine is performed. Simulation of clinical trials, optimization of surgical techniques and medical device designs, and additive manufacturing, are among the most compelling and impactful areas that may benefit from this technology.