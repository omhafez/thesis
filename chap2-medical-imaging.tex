\chapter{The Second Chapter}

\section{Overall Appearance}
%
Do not bind or punch holes in the thesis or dissertation; the document will be accepted as loose pages only.

The general appearance of the manuscript is important, not only for Library use at UC Davis, but also since the copy which doctoral students submit will be sent to UMI Dissertation Publishing to be copied on request for scholars around the world. For this reason, correction of typographical errors should be made with care. Pages with interlineations, crossed out words or letters, strike-overs and unsightly corrections will be rejected and the candidate will be required to replace them. The use of white correction fluid or white correction carbon IS NOT ACCEPTABLE. If you photocopy your dissertation or thesis from laser/copier paper to the accepted 25\% cotton bond thesis paper, please make certain each page is clear, with no dark edges.

\subsection{Tables, Graphs and Captions}
%
Charts and tables may be placed horizontally or vertically, but in either case must fit within the required margins. It may be necessary to use a reducing copier in order to achieve this. If necessary, wide tables, charts, and figures can be placed sideways. Figures may be embodied in the text or take up a full page. Each figure or table must be numbered consecutively (do not renumber each chapter unless you include chapter numbers, e.g., Fig.~1.1, Fig.~2.1, etc.) and should have a caption. With full page figures, this can be done on the facing caption page. Such a caption page must be the mirror image of a normal page, i.e., the wider margin is on the right and the number in the upper left hand corner. The caption itself should be centered and should always be on a page by itself, not on the back of a preceding page.

NOTE: If your figures or charts are placed horizontally on the page, your page number must still be in the upper right corner when the page is viewed in the manuscript. Pagination must be consistent.

\subsection{Photographs, Illustrations and Maps}
%
Plates or figures should be the original photographs or drawings, or photographic reproductions of the originals. All reproductions must be clean and distinct. Photographs should be positive prints; line drawings or tabular material reproduced by photography should be positives. If color is important, special markings should be used to identify color variations. If a plate is smaller than the standard $8.5\times 11$~inch page, it should be attached to a blank page of standard size with dry mounting tissue, white glue or a permanent spray adhesive. If white glue is used, the complete perimeter should be covered. Do not use rubber cement, tapes, mucilage, or photo-corners to attach plates. These are cited because they are well known and are not permanent processes; this is not a complete list of unacceptable methods. You should inquire about the acceptability of any other method that you may consider using. If a digital graphic is used and printed on photographic paper, this is acceptable if used only in this manner. Pagination must be consistent.

\subsection{Oversized Material}
%
Tables, charts, and plates that are larger than $8.5\times 11$~inches in size must not be folded, since they will have to be folded in a special manner for binding. They should be submitted in a cardboard cylinder, available at the UC Davis Bookstore or any stationery store, and must also be numbered with Arabic numbers.

Extremely oversized material should also be submitted in a cardboard cylinder. These oversized sheets should not be assigned page numbers, but your name and title of your dissertation or thesis should appear on each sheet in the lower right hand corner. The Library will fold this material for placement in a special pocket when the manuscript is bound.

\subsection{Using Published Material}
%
If approved by the thesis or dissertation committee, reports of research undertaken during graduate study at UC Davis which have been published may be accepted in printed form as all or part of the master's thesis or doctoral dissertation. If you are not the sole or first author of the published material submitted, the use of co-authored materials must be approved by the department or graduate group concerned.

The pages of the published material must have the same margins and type of paper as specified in these instructions which may necessitate reduction of the material. Note that enlargement of the materials to meet standard margin requirements is not needed. When using reprints, page numbers of the reprint should be removed and replaced with numbers corresponding to the position of the reprint within the dissertation. Each chapter may have an abstract of its own. For doctoral students there must be a general abstract covering the entire dissertation.

If the papers or articles to be submitted are not yet in published form, they must conform to the specifications of these instructions.

\subsection{Computer-produced Dissertations and Theses}
%
Manuscripts done on a computer or word processor should be produced either on a hard-copy terminal or phototypesetter. These are just a few of the software packages which will operate within the format described in these instructions:

\begin{itemize}
  \item \TeX and \LaTeX
  \item Deltagraph
  \item Microsoft Word
  \item Wordperfect
\end{itemize}

Laser and other high quality printers are acceptable.


\subsection{Methods for Printing Final Copies}
%
Graduate Studies will accept a thesis or dissertation prepared by any one of a number of processes provided that it meets all of the following conditions:

\begin{itemize}
  \item it must be a legible, clean, clear copy;
  \item it must be photographically reproducible;
  \item the ink must be permanent;
  \item a minimum of 20~lb.\ 25\% cotton bond paper is required. (Please select paper with a watermark so that it is apparent that the paper meets the minimum requirements.)
\end{itemize}

The following are acceptable processes: laser printer, ink jet printer, typed, Xerox, etc. This is not a complete list of acceptable processes and you should consult Graduate Studies concerning any process you are planning to use. Even though a process is acceptable on other grounds, the copy submitted will be rejected if it is not clear and legible. If you are uncertain about the acceptability of any copy you are planning to submit, please consult Graduate Studies.

The following processes are NOT acceptable: Ditto, Hectograph, Thermofax, blueprint, dot matrix printer. These are mentioned because they are widely used for other purposes; this is not a complete list of unacceptable processes.

Before submitting the copy to Graduate Studies, check to be sure that every page has been copied, numbered correctly, and is in the right order.

The ORIGINAL copy of the title page containing the original signatures of your committee must, in all cases, be submitted with the thesis or dissertation. It is preferable that the title page be printed on 25\% cotton bond paper.

\subsection{Number of Copies}
%
One unbound approved copy must be filed in Graduate Studies for later deposit in the University Library. You should consult your graduate staff person, graduate adviser, and committee chair to determine whether additional copies are required by your department or graduate group. You are responsible for providing any additional copies in these cases. Two places that offer binding are Cal Na Binding in Sacramento (916-447-4355) and Navin's Copy Shop in Davis (530-758-2311) which contracts with Cal Na.

\subsection{Dates of Filing}
%
Check the calendar for deadlines for filing the master's thesis or the doctoral dissertation with the committees in charge and with Graduate Studies. Deadlines are also announced each year in the Class Schedule and Registration Guide and the General Catalog. The deadline for filing with your committee is a recommended deadline to allow time for making revisions. The deadline for filing with Graduate Studies is firm.

It is important to bring all documents, forms and supplies with you when you file your thesis or dissertation. Please review the checklist for master's or doctoral students prior to your appointment.

\subsection{Title Page}
%
Graduate Studies does not supply the title page. You must prepare your title page in accordance with the sample. The title page is to be signed by all members of your committee when they have approved the thesis or dissertation. Only the original title page will be accepted with the thesis or dissertation.

\subsection{Dissertation Abstracts}
%
You must provide two different forms of your abstract. The first is submitted separately from the dissertation to University Microfilms International (UMI) Dissertation Publishing. The other format follows the regular formatting guidelines and is submitted as part of your dissertation.

The abstract that is submitted to UMI must be formatted as shown in the example here. The body of the abstract cannot exceed 350 words. It should be in typewritten form, double-spaced, and on bond paper. It is important to write an abstract that gives a clear description of the content and major divisions of the dissertation, since UMI will publish the abstract exactly as submitted. Students completing their requirements under Plan A should provide extra copies of the typed summary for use by the dissertation committee during the examination.

The abstract submitted as part of your dissertation, in the introductory pages, does not have a word limit. It should follow the same format as the rest of your dissertation (1.5 inch left margin, double-spaced, consecutive page numbering, etc.).

\subsection{University Microfilms International}
%
As a doctoral candidate, you will sign an agreement with University Microfilms International/Proquest Information and Learning Company (PQIL) for the microfilming of the dissertation or the printing of the abstract on the dissertation abstract database. Forms are available in the Office of Graduate Studies and should be filled out and submitted to Graduate Studies with the dissertation. You will be given the option to apply for copyright, but it is not required. If you would like to copyright your dissertation, please see the section on copyright and publication.

\subsection{Diploma}
%
When you file your thesis or dissertation, you will receive a temporary certificate that states you have completed all the requirements for your program and the official conferral date of your degree. This certificate may be given to your employer for proof of degree until the Registrar's Office issues an official transcript or diploma. You must complete a form to request your transcript or diploma. Official transcripts normally are available two months after the official degree conferral date; diplomas normally are available four months after this date.

\subsection{Copyright and Publication}
%
The copyright law of the United States is quite complex. The information contained in this section is only a general guide--more detailed information must be obtained from other sources. Whether or not you copyright your thesis or dissertation, you retain the right to publish all or any part of it by any means at any time, except for reproduction from a negative microfilm as described in the agreement form with University Microfilms International, if you signed such an agreement. Should you decide to copyright your thesis or dissertation, you must include a separate copyright page after the title page; do not use a page number on the copyright page. By adding this copyright notice, which should be included in all copies you distribute, you have copyrighted your thesis or dissertation. At this point you have several options:

If you are a doctoral student, you may have the copyright registered for you by roquest Information and Learning Company (PQIL). Along with the UMI Doctoral Dissertation Agreement form, you will need to submit a U.S. Postal money order in the amount of \$65 (no personal checks), payable to PQIL, to cover the copyright fee and the cost of two positive microfilm copies which will be filed with the Copyright Office.

You may register the copyright yourself by submitting to the Registrar of Copyrights the appropriate application form, a filing fee and one or two copies of the work. In order to have full protection against infringement, this should be done as soon as possible. Information and forms can be obtained from the Registrar of Copyrights, Library of Congress, Washington D.C. 20559.

You may choose to copyright your thesis or dissertation by adding the copyright notice, submitting a copy to the Registrar of Copyrights, but not registering it. (Federal copyright law requires that copies of all works published with notices of copyright be deposited with the Library of Congress, even if the copyright is not registered). However, to protect your rights in a copyright dispute and in order to be compensated for damages caused by infringement, your copyright must be registered.

\subsection{Permissions}
%
Since the depositing of your thesis or dissertation in the University Library and/or its being made available by UMI/PQIL may constitute a form of publication, you may have to obtain permission to use (or quote) copyrighted material, such as that in most journal articles or books. It is the author (i.e., you) who is responsible in the matter of copyrighted materials. The agreement form which you sign with University Microfilms specifically absolves them of any such responsibility.

If you quote extensively from a particular author, especially in fields such as fiction, drama, criticism, or poetry, or if copyrighted maps, charts, statistical tables, or similar materials have been reproduced, you must write the copyright owner(s), describe the use which you are making of the materials, and request permission to use it in the dissertation or thesis.

For your protection, a statement listing such materials should be included in the acknowledgements of the dissertation or thesis. The statement should inform the reader (1) that permission has been granted for their use, and (2) the source of the permission.

\subsection{Filing Fee}
%
The Filing Fee was established expressly to assist those students who have completed all requirements for degrees except filing theses or dissertations and/or taking final examinations (master's comprehensive exams or doctoral final examinations) and are no longer using University facilities. The Filing Fee is a reduced fee paid in lieu of registration fees. It is assessed only once and must be paid to the Cashier's Office prior to submission of the form to Graduate Studies. Filing Fee status restrictions (more restrictions are noted on the application instruction sheet):
%
\begin{itemize}
  \item You may not be using University facilities;
  \item You cannot be using faculty time other than the time involved in the final reading of the thesis or dissertation or in holding final examinations;
  \item You are not eligible to hold any academic appointment title for more than 1 quarter;
  \item You cannot hold a fellowship or receive financial aid.
\end{itemize}

If you are eligible to use the Filing Fee procedure, you should complete a Filing Fee application, obtain the signature of the Graduate Adviser and your Committee chair, and return the application to Graduate Studies before you stop registering. The Filing Fee must be paid prior to submitting the application to Graduate Studies.

Original (initial) filing fee deadlines adhere to registration deadlines. For example, if you were approved for one quarter of filing fee, you would be eligible to submit your dissertation or thesis up to the last day of late registration for the following quarter. If you do not submit by that date your filing fee status will expire and you would need to secure an extension from your program and from Graduate Studies. Filing fee extensions end the date noted on the petition. Make sure your filing fee is current before you submit your dissertation or thesis.
